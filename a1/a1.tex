\documentclass[a4paper, 12pt]{article}

\usepackage[left=2cm, right=2cm, top=2cm]{geometry}
\usepackage{color}
\usepackage{graphicx}
\usepackage{float}

% Libraries for drawing search tree
\usepackage{tikz}
\usetikzlibrary{trees}
\tikzstyle{level 1}=[level distance=3.5cm, sibling distance=6cm]
\tikzstyle{level 2}=[level distance=3.5cm, sibling distance=6cm]
\tikzstyle{bag} = [align=center, style={rectangle, draw=black}]
\tikzstyle{nope} = [label=below:\Large{{\color{red}X}}]
\tikzstyle{circ} = [align=center, style={circle, draw=black}]

\usepackage{adjustbox}
\usepackage{forest}
\usepackage{tikz-qtree,tikz-qtree-compat}


\begin{document}
\pagenumbering{roman}
\title{
		\textbf{Student Name:} Tevin Achong\\
		\textbf{Student ID:} 816000026\\
		\textbf{Student Name:} Name 2\\
		\textbf{Student ID:} ID2\\						
		\textbf{Course Code:} COMP3608\\
		\textbf{Course Title:} Intelligent Systems\\
		\textbf{Assignment:} 1
		\date{March 1st, 2020}
}
\maketitle

\newpage
\pagenumbering{arabic}

\begin{center}
	\textbf{Question 1}
\end{center}
For this question, we let $w_i$ indicate the $ith$ woman, and $m_i$ indicate the $ith$ man. The notation $w_im_i$ indicates that the $ith$ woman marries the $ith$ man. For example, $w_2m_3$ would indicate that the 2nd woman marries the 3rd man. The following search tree is derived for the Hall's Marriage Problem with:
\begin{center}
$w_0$ having preferences $m_0, m_1, m_2$\\
$w_1$ having preferences $m_0, m_3, m_5$\\
$w_2$ having preferences $m_2, m_5$
\end{center}


\begin{adjustbox}{width=\linewidth}
\begin{tikzpicture}[sloped]
	\node[bag] {{\color{blue}0}\\ $w_0m_0$}
    	child {
			node[bag, label=below:\Large{{\color{red}X}}] {{\color{blue}1}\\$w_1m_0$}
		}		
		child {
			node[bag, label=below:\Large{{\color{red}X}}] {{\color{blue}2}\\$w_1m_1$}
		}
		child {
			node[bag, label=below:\Large{{\color{red}X}}] {{\color{blue}3}\\$w_1m_2$}
		}
		child {
			node[bag] {{\color{blue}4}\\$w_1m_3$}
				child {
					node[bag, label=below:\Large{{\color{red}X}}] {{\color{blue}5}\\$w_2m_0$}
				}
				child {
					node[bag, label=below:\Large{{\color{red}X}}] {{\color{blue}6}\\$w_2m_1$}
				}
				child {
					node[bag, label=below:\Large{Solution Found}] {{\color{blue}7}\\$w_2m_2$}
				}
		};
\end{tikzpicture}
\end{adjustbox}

Explanation:
\begin{enumerate}
\item
Node 0: We try $w_0m_0$. Since $m_0$ is in the set of preferences of $w_0$ and no $w$ has yet married $m_0$ we continue down this path.
\item
Node 1: We try $w_1m_0$. This cannot work because even though $m_0$ is in the set of preferences of $w_1$, $w_0$ has already married $m_0$, so we abort this path.
\item
Node 2: We try $w_1m_1$. This cannot work since $m_1$ is not in the set of preferences of $w_1$, so we abort this path.
\item
Node 3: We try $w_1m_2$. This cannot work since $m_2$ is not in the set of preferences of $w_1$, so we abort this path.
\item
Node 4: We try $w_1m_3$. Since $m_3$ is in the set of preferences of $w_1$ and no $w$ has yet married $m_3$ we continue down this path.
\item
Node 5: We try $w_2m_0$. This cannot work since $m_0$ is not in the set of preferences of $w_2$, and $w_0$ has already married $m_0$, so we abort this path.
\item
Node 6: We try $w_2m_1$. This cannot work since $m_1$ is not in the set of preferences of $w_2$, so we abort this path.
\item
Node 7: We try $w_2m_2$. Since $m_2$ is in the set of preferences of $w_2$ and no $w$ has yet married $m_2$ we accept this pair. And since all women have been paired with a man, this is our solution.
\end{enumerate}

\textbf{Solution:} $w_0m_0$, $w_1m_3$, $w_2m_2$.


\end{document}